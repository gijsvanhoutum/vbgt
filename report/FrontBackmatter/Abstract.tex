\phantomsection
\addcontentsline{toc}{chapter}{Abstract}

\chapter*{Abstract}

% motivation
Conventional manufacturing processes lack the free form capabilities additive manufacturing (AM) has. Achieving accuracy and short production time are however challenging to achieve with AM. If these problem could be resolved, AM could replace the conventional methods to mass produce more complex parts.

% problem statement
Feedback control of deposited material dimensions in soft material AM extrusion processes could improve the accuracy of parts. Measuring the deposition is possible using a triangulation based vision system consisting out of a camera and a laser-diode. Algorithms exist to extract geometric features such as height and width but lack flexibility and robustness against noise. 

% approach
A literature review on algorithm design for triangulation based vision systems and an experimental low-quality vision system are used to identify the limitations in current algorithms and their applicability to low-quality vision hardware.
 
% results
This research produces a new algorithm able to detect height and width of deposition material for addtive manufacturing processes. It is able to operate under pixel saturation and inhomogenity of the laser line.  It is flexible since only one parameter is needed for the algorithm.


\cleardoublepage
